\documentclass[letter,12pt]{article}

\usepackage[T1]{fontenc}
\usepackage[utf8]{inputenc}
\usepackage[francais]{babel}
\usepackage{geometry,listings}
\geometry{margin=2.5cm}
\usepackage{hyperref}

\title{Création et mise en ligne de site web avec GitHub : notes et remarques pour commencer}
\author{Pauline Hubert}


\begin{document}
	\maketitle
	
	Quelques ressources utiles : 
	\begin{itemize}
		\item Tutoriel sur GitHub Pages : \url{https://gist.github.com/TylerFisher/6127328}
		\item Tutoriel sur GitHub Pages : \url{http://jmcglone.com/guides/github-pages/}
		\item Markdown : \url{https://github.com/adam-p/markdown-here/wiki/Markdown-Cheatsheet}
		\item Markdown : \url{https://guides.github.com/features/mastering-markdown/}
		\item Jekyll : \url{https://jekyllrb.com/}\\
	\end{itemize}
	
	Quelques remarques pour commencer.
	
	\begin{enumerate}
		\item GitHub permet de faire des sites web statiques. 
		
		\item Si ce n'est pas déjà fait, se créer un compte GitHub. Créer un dépôt qui porte le nom \texttt{username.github.io} pour que github sache que cela va être un site web. L'addres url du site web sera alors \texttt{https://username.github.io} .
		
		\item Le dépôt va contenir les fichiers (html, css, markdown) de notre site web. Le fichier de la page d'accueil doit s'appeler \texttt{index.md} ou \texttt{index.html}. Si GitHub ne trouve pas de tel fichier, il va utiliser le readme comme page d'accueil. 
		
		\item GitHub utilise \texttt{jekyll} pour généré le site web à partir des fichiers dans le dépôt. On peut donc utiliser les options de jekyll, y compris choisir un thème prédéfini. 
		
		Pour cela, dans GitHub, aller dans \texttt{settings} aller à la section GitHub Pages. Le choix d'un thème va créer un fichier \texttt{\_config.yml} qui pourra ensuite être compléter pour configurer d'autres choses. 
		
		\item Conseil : aller lire le readme du thème choisi pour obtenir des conseils sur commencer personnaliser le thème. 
		
		\item Créer ensuite des fichiers en markdown pour le contenu.
		
		\item Il est aussi possible de tout faire à la main en html et css. GitHub va être capable de générer aussi le site web. 
		
		\item Je pense qu'il est possible de combiner les deux mais je ne sais pas trop comment. 
		
		\item Pour visualiser le site web, après avoir commité, attendre quelques secondes puis aller à l'adresse \texttt{https://username.github.io} pour voir directement le site web. 
		
		\item Remarque : On peut aussi créer un dépôt  \texttt{new\_repo.github.io}, puis aller dans la section GitHub Pages de \texttt{settings} pour choisir l'option permettant de transformer le dépôt en site web. L'adresse sera alors \url{http://username.github/new\_repo.io}. 
	\end{enumerate}

	L'adresse de mon site web : \url{http://phubert.github.io}
	
	Mon dépôt git : \url{https://github.com/phubert/phubert.github.io.git}
\end{document}

