\documentclass[letter,12pt]{exam}
\usepackage[T1]{fontenc}
\usepackage[utf8]{inputenc}
\usepackage[francais]{babel}
\usepackage{amsfonts,amsmath,amssymb,amsthm} % packages à inclure pour les symboles et environnements mathématiques
\usepackage{geometry} % controle des marges
\geometry{margin=2.5cm}
\usepackage{multicol} % texte en colonnes
\usepackage{graphicx} % inclusion d'images
\usepackage{float} % placement des images
\usepackage{aurical}
\usepackage{fancybox}

%% Déclaration des opérateurs mathématiques et des nouvelles commandes
\newcommand{\norme}[1]{\left|\!\left| #1 \right|\!\right|}
\newcommand{\vecteur}[1]{\overrightarrow{#1}}


%%- 1 ère page
\title{{\Large{\textbf{Devoir 3}}}} 
\date{Mercredi 13 novembre 2019}

%%% Remplacer ici par votre prénom, nom et code permanant
\author{PRENOM NOM \\ CODE PERMANENT}


% En-tête
\pagestyle{headandfoot}
\runningheadrule
\firstpageheader{MAT0600 - Algèbre linéaire et géométrie vectorielle}{}{Automne 2019}
\runningheader{MAT0600 - Algèbre linéaire et géométrie vectorielle}{}{Automne 2019}
\firstpagefooter{}{\thepage\ / \numpages}{}
\runningfooter{}{\thepage\ / \numpages}{}


\begin{document}

\maketitle 
\thispagestyle{headandfoot}

\hrulefill

\qformat{\bfseries Exercice \thequestion: \thequestiontitle \hfill \totalpoints\ points}
\pointsinrightmargin
\printanswers


\begin{questions}

\vspace*{.5cm}
\titledquestion{La chasse au trésor} ~

\begin{center}
	{\Fontlukas{\large{Bienvenue dans le monde merveilleux de la géométrie vectorielle! 
					
	Quelque part sur ces terres inexplorées se trouve un fabuleux trésor. 
				
	Pour le trouver, vous disposez des indications suivantes. 
	
	Bon courage! }}}
\end{center}

\vspace{.5cm}

On fixe une base orthonormée de l'espace.
Vous démarrez votre aventure au point de départ $D = (2,3,1)$ et vous avez en votre possession une carte et une série d'instructions : 

\begin{center}
	\ovalbox{$F = (0,3,3)$ \quad {\Fontlukas{ et }} \quad $L = (4,2,4)$}
\end{center}

\vspace{.2cm}  

\begin{parts}
	\part[2] Pour bien vous repérer sur la carte, vous avez besoin de connaître la distance entre vous et l'origine $O= (0,0,0)$.
	Calculer la longueur $\norme{\vecteur{OD}}$ du vecteur $\vecteur{OD}$.
	\begin{solutionorbox}
		%%%% ECRIRE LA SOLUTION ICI %%%%
	\end{solutionorbox}
	
	\part[5] Maintenant que vous avez une meilleure idée de la taille de la carte, vous devez traverser la forêt devant vous et vous rendre au point $F$.
	Pour cela, calculez l'angle entre les vecteurs $\vecteur{OD}$ et $\vecteur{OF}$ ainsi que les coordonnées du vecteur $\vecteur{DF}$.
	\begin{solutionorbox}
		%%%% ECRIRE LA SOLUTION ICI %%%%
	\end{solutionorbox}
	
	\part[6] Vous devriez maintenant être au point $F$. La prochaine étape de votre quête va vous mener au point $L$. Mais attention, pour ne pas risquer d'être pris dans les sables mouvants, vous devez passer par le point $P_F$ le projeté orthogonal de $F$ sur le vecteur $\vecteur{OL}$. Quelles sont les coordonnées de $P_F$ ? 
	
	\textit{Indication : les coordonnées de $P_F$ sont données par $\vecteur{OP_F}$ le projeté orthogonal de $\vecteur{OF}$ sur $\vecteur{OL}$.} 
	\begin{solutionorbox}
		%%%% ECRIRE LA SOLUTION ICI %%%%
	\end{solutionorbox} 
	
	\part[3] Vous êtes en route vers le point $L$ mais sur votre chemin, vous apercevez un grand lac que vous ne pouvez pas contourner sans risquer de vous retrouver dans les sables mouvants.
	Vous allez devoir traverser le lac. Vous décidez donc de construire un plan sur lequel vous pourrez marcher. Donner des équations paramétriques du plan $\mathcal{P}_1$ passant par les points $F$, $P_F$ et $L$.
	\begin{solutionorbox}
		%%%% ECRIRE LA SOLUTION ICI %%%%
	\end{solutionorbox}
	
	\part[9] Après avoir construit le plan $\mathcal{P}_1$, vous pouvez enfin traverser le lac et arriver au point $L$.
	Quand vous arrivez au point $L$, un elfe vous donne l'instruction suivante.
	\begin{center}
		\ovalbox{{\Fontlukas{ Le prochain indice se trouve quelque part à l'intersection des plans $\mathcal{P}_1$ et $\mathcal{P}_2$.}}}
	\end{center}
	Le plan $\mathcal{P}_2$ est le plan passant par $D$ et de vecteur normal $\vecteur{n_2} = (1,0,1)$. Calculer une équation de la droite $\mathcal{D}$ d'intersection de ces deux plans. 
	\begin{solutionorbox}
		%%%% ECRIRE LA SOLUTION ICI %%%%
	\end{solutionorbox}
	
	\part[3] Avant de disparaître, l'elfe vous offre une potion magique qui vous permet de voler directement jusqu'à la droite $\mathcal{D}$ mais vous devez d'abord déterminer la distance entre vous et la droite $\mathcal{D}$ pour savoir quelle quantité de potion vous avez besoin. L'elfe vous rappelle que vous vous trouvez présentement au point $L$.
	\begin{solutionorbox}
		%%%% ECRIRE LA SOLUTION ICI %%%%
	\end{solutionorbox} 
	
	\part[3] Vous êtes maintenant sur la droite $\mathcal{D}$ mais la fée qui doit vous donner votre prochaine instruction a mélangé ses parchemins et ne sait plus à quel point vous devez maintenant vous rendre. Trouvez parmi les points suivants lequel appartient à la droite $\mathcal{D}$ et rendez vous à ce point. 
	
	\begin{center}
		\ovalbox{$A = (6,12,-3)$} \hspace*{.8cm} \ovalbox{$B = (-6,12,9)$} \hspace*{.8cm}  \ovalbox{$C = (-2,3,2)$}
	\end{center}

	\begin{solutionorbox}
		%%%% ECRIRE LA SOLUTION ICI %%%%
	\end{solutionorbox}
	
	\part[2] Votre quête touche à sa fin! Le grand magicien de la géométrie vectorielle détient les coordonnées du trésor mais il ne vous les donnera que si vous répondez correctement à son énigme. 
	
	\begin{center}
	\ovalbox{\parbox{.8\textwidth}{ \centering {\Fontlukas{ Pourquoi est-ce que je peux affirmer sans calcul que le plan de vecteur normal $\vecteur{OP_F}$ et le plan de vecteur normal $\vecteur{OL}$ sont parallèles? }}}}
	\end{center}
	
	\begin{solutionorbox}
		%%%% ECRIRE LA SOLUTION ICI %%%%
	\end{solutionorbox}
	
	\part[2] Bravo, vous avez réussi l'épreuve du grand magicien de la géométrie vectorielle et il vous a révélé que le trésor se trouve au point de coordonnées $T = (42,42,42)$. Vous vous y rendez pour y découvrir le trésor.  
	\begin{center}
	\ovalbox{{\Fontlukas{ Félicitations, vous gagnez $2$ points pour avoir essayé chaque question de cet exercice! }}}
	\end{center}
\end{parts}


\vspace*{.5cm}
\titledquestion{Matrices inverses}[15] ~

Pour chacune des matrices suivantes, vérifier qu'elle est inversible et si oui donner son inverse. 

\begin{multicols}{3}
	\begin{parts}
		\part $A = \left(\begin{array}{rr}
		1 & 2 \\
		2 & -1
		\end{array}\right)
		$
		\part $B = \left(\begin{array}{rrr}
		1 & 0 & 1 \\
		3 & -1 & 4 \\
		0 & 1 & -2
		\end{array}\right)
		$
		\part $C = \left(\begin{array}{rrr}
		1 & 1 & 0 \\
		-2 & -2 & 0 \\
		1 & 2 & -1
		\end{array}\right)$
	\end{parts}
\end{multicols}

\begin{solutionorbox}
	\begin{parts}
		\part %%%% ECRIRE LA SOLUTION ICI %%%%
		\part %%%% ECRIRE LA SOLUTION ICI %%%%
		\part %%%% ECRIRE LA SOLUTION ICI %%%%
	\end{parts}
\end{solutionorbox}


\end{questions}
\end{document}




