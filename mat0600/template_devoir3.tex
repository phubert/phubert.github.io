\documentclass[letter,12pt]{exam}
\usepackage[T1]{fontenc}
\usepackage[utf8]{inputenc}
\usepackage[francais]{babel}
\usepackage{amsfonts,amsmath,amssymb,amsthm} % packages à inclure pour les symboles et environnements mathématiques
\usepackage{geometry} % controle des marges
\geometry{margin=2.5cm}
\usepackage{multicol} % texte en colonnes
\usepackage{graphicx} % inclusion d'images
\usepackage{float} % placement des images
\usepackage{aurical}
\usepackage{fancybox}

%% Déclaration des opérateurs mathématiques et des nouvelles commandes
\newcommand{\RR}{\mathbb{R}}


%%- 1 ère page
\title{{\Large{\textbf{Devoir 3}}}} 
\date{Mercredi 27 novembre 2019}

%%% Remplacer ici par votre prénom, nom et code permanant
\author{PRENOM NOM \\ CODE PERMANENT}


% En-tête
\pagestyle{headandfoot}
\runningheadrule
\firstpageheader{MAT0600 - Algèbre linéaire et géométrie vectorielle}{}{Automne 2019}
\runningheader{MAT0600 - Algèbre linéaire et géométrie vectorielle}{}{Automne 2019}
\firstpagefooter{}{\thepage\ / \numpages}{}
\runningfooter{}{\thepage\ / \numpages}{}


\begin{document}

\maketitle 
\thispagestyle{headandfoot}

\hrulefill

\qformat{\bfseries Exercice \thequestion: \thequestiontitle \hfill \totalpoints\ points}
\pointsinrightmargin
\printanswers

\begin{questions}
	
	\vspace*{.5cm}
	\titledquestion{Fibomatrice}[5] ~

	Calculer le déterminant de $M$. $$M = \left(\begin{array}{rrrrrr}
	1 & 1 & 2 & 3 & 5 & 8 \\
	1 & 2 & 3 & 5 & 8 & 13 \\
	2 & 3 & 5 & 8 & 13 & 21 \\
	3 & 5 & 8 & 13 & 21 & 34 \\
	5 & 8 & 13 & 21 & 34 & 55 \\
	8 & 13 & 21 & 34 & 55 & 89
	\end{array}\right)
	.$$
	(\textit{Indication : Il est possible d'effectuer le calcul très rapidement.})
	
	\begin{solutionorbox}
		%%%% ECRIRE LA SOLUTION ICI %%%%
	\end{solutionorbox}
	
	\vspace*{.5cm}
	\titledquestion{Deltaminants} ~
	
	On pose 
	
	$$\Delta_2 = \begin{vmatrix}
	2 & -1 \\ -1 & 2
	\end{vmatrix} \hspace{1cm}
	\Delta_3 = \begin{vmatrix}
	2 & -1 & 0 \\ -1 & 2 & -1 \\ 0 & -1 & 2
	\end{vmatrix} \hspace{1cm}
	\Delta_4 = \begin{vmatrix}
	2 & -1 & 0 & 0 \\ -1 & 2 & -1 & 0 \\ 0 & -1 & 2 & -1 \\ 0 & 0 & -1 & 2
	\end{vmatrix}$$
	
	et pour tout $n \geq 5$, $\Delta_n = \begin{vmatrix}
	2 & -1 & 0 & \cdots & \cdots & 0 \\
	-1 & 2 & -1 & \ddots &  & \vdots \\
	0 & -1 & 2 & -1 & \ddots & \vdots \\
	\vdots & \ddots & \ddots & \ddots & \ddots & 0 \\
	\vdots & & \ddots & -1 & 2 & -1\\
	0 & \cdots & \cdots & 0 & -1 & 2 
	\end{vmatrix}$ \\
	
	\begin{parts}
		\part[5] Calculer $\Delta_2$, $\Delta_3$ et $\Delta_4$.
		\begin{solutionorbox}
			%%%% ECRIRE LA SOLUTION ICI %%%%
		\end{solutionorbox}
		\part[5] Montrer que pour tout $n \geq 4$, $\Delta_{n} = 2\Delta_{n-1} - \Delta_{n-2}$. 
		\begin{solutionorbox}
			%%%% ECRIRE LA SOLUTION ICI %%%%
		\end{solutionorbox}
	\end{parts}
	
	\vspace*{.5cm}
	\titledquestion{Règle de Cramer}[10] ~
	
	Résoudre le système d'équations linéaires suivant en utilisant la règle de Cramer.  
	$$\left(\begin{array}{rrr}
	1 & 2 & 4 \\
	0 & -2 & -1 \\
	1 & 1 & 3
	\end{array}\right)
	\left(\begin{array}{r}
	 x_{1} \\
	 x_{2} \\
	 x_{3}
	 \end{array}\right) = \left(\begin{array}{r}
	 -9 \\
	 -2 \\
	 -4
	 \end{array}\right).$$
	
	\begin{solutionorbox}
		%%%% ECRIRE LA SOLUTION ICI %%%%
	\end{solutionorbox}
	
	\vspace*{.5cm}
	\titledquestion{Matrices inverses}[10] ~
	
	Déterminer si les matrices suivantes sont inversibles et si oui donner leur inverse en utilisant la matrice adjointe. 
	
	$$A=\left(\begin{array}{rrr}
	1 & 2 & 4 \\
	0 & -2 & -1 \\
	1 & 1 & 3
	\end{array}\right) \qquad B=\left(\begin{array}{rr}
	-1 & 7 \\
	2 & -3
	\end{array}\right)
	 \qquad C=\left(\begin{array}{rrr}
	 1 & 0 & 2 \\
	 8 & -3 & 1 \\
	 -12 & 6 & 6
	 \end{array}\right)
	 $$
	
	\begin{solutionorbox}
		%%%% ECRIRE LA SOLUTION ICI %%%%
	\end{solutionorbox}
	
	\vspace*{.5cm}
	\titledquestion{Sous-espaces vectoriels} ~
	Déterminer si chacun des sous-ensembles suivants est un sous-espace vectoriel. 
	
	\begin{parts}
		\part[5] $H_1 = \{ (x, -x^2); x \in \RR \}$ est-il un sous-espace vectoriel de $\RR^2$ ?
		\begin{solutionorbox}
			%%%% ECRIRE LA SOLUTION ICI %%%%
		\end{solutionorbox}
		\part[5] $H_2 = \{ (a, b-a, 2b); a,b \in \RR \}$ est-il un sous-espace vectoriel de $\RR^3$ ?
		\begin{solutionorbox}
			%%%% ECRIRE LA SOLUTION ICI %%%%
		\end{solutionorbox}
		\part[5] $H_3 = \{ (x,y,z) \in \RR^3 \text{ tels que } x-y=z\}$ est-il un sous-espace vectoriel de $\RR^3$ ?
		\begin{solutionorbox}
			%%%% ECRIRE LA SOLUTION ICI %%%%
		\end{solutionorbox}
	\end{parts}
\end{questions}

\end{document}




